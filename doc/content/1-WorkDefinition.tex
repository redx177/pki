\chapter{Beschreibung der Aufgabe}

\section{Aufgabenstellung}

\subsection{Ausgangslage}
Im Fach \Gls{pkiLabel} wird der sichere Austausch von Nachrichten und Schlüsseln behandelt. Diese Seminararbeit bezieht sich auf das \Gls{pkiLabel} Fach. Es wurden verschiedene Angriffszenarien in Bezug auf die Sicherheit vorgestellt. Dieses Seminar befasst sich mit Phishing und Social Engineering.
Phishing und Social Engineering sind keine typischen Angriffszenarien, wie man sie sich vorstellt. Normalerweise werden Angriffe von einem weit entfernten Computer durchgeführt. Bei diesem Thema tritt der Angreifer direkt in Erscheinung. Er interagiert mit dem Angegriffenen und versucht durch geschickte Techniken an Daten oder Zugriffsrechte zu gelangen. 

\subsection{Ziele der Arbeit}
Ziel der Arbeit ist es, Social Engineering sowie die unterart Phishing vorzustellen. Es werden Techniken erläutert, sowie Massnahmen wie man sich dagegen schützen kann. Es gibt auch diverse Interessante Beispiele die in der Arbeit aufgezeigt werden. 

\subsection{Aufgabenstellung}
Es soll eine Arbeit im Umfang von ca. 15-30 Seiten erstellt werden. Das Thema ist Phishing und Social Engineering. Das Papier soll die beiden Themen erläutern und die Unterschiede aufzeigen.
Zum Schluss gibt es noch eine Präsentation die den Inhalt der Arbeit für den Dozenten sowie den Rest der Klasse anschaulich zusammenfasst.

\subsection{Erwartete Resultate}
Das erwartete Resultat der Arbeit ist eine Einführung in das grosse Thema des Social Engineering sowie Phishing. Die Arbeit soll aufzeigen, was diese Antriffstechniken sind, welche Techniken verwendet werden und was die Gefahr dabei ist. Auch Teil der Arbeit sind Verteidigungsmassnahmen gegen die Techniken.
Die Präsentation soll die Arbeit für die Mitstudenten sowie den Dozenten anschaulich und unterhaltsam zusammenfassen.