\chapter{Schlussfolgerung}

Social Engineering ist eine Angriffsform, die weniger technischer- dafür psychologischer Natur ist. Der Mensch rückt ins Zentrum des Angriffes, nicht die Maschine. Grundlagen für einen Angriff sind schnell erlernt. Doch um erfolgreich zu sein muss man die vorgestellten Techniken lernen und verinnerlichen, sowie selbstverständlich sehr viel üben. Es wird auch Mut vorausgesetzt, denn als Angreifer tritt man aus dem Schatten in das Licht. 

Das Social Engineering ist dabei sehr viel sicherer als zunächst angenommen wird. Viele Attacken bleiben unerkannt, da die Opfer diese gar nicht bemerken. Denn das Sicherheitsbewusstsein für eine solche Attacke ist noch unausgereift.

Phishing wirft einen Köder in einen grossen See voller möglichen Opfer und hofft, dass wenige anbeissen. Die Angriffe können von sehr primitiv bis zu sehr ausgeklügelt reichen. Meistens sind sie jedoch für ein geschultes Auge leicht erkennbar. Aber es wird immer ein Fisch geben welcher den Köder schluckt.

\section{Persönliches}
Zur Einarbeitung in das Social Engineering benötigte ich sehr viel Zeit. Das Thema beginnt sehr trocken und hat wenig technische Aspekte. Je tiefer ich jedoch in die Materie eindrang, desto mehr packte mich das Interesse. Ich konnte sehr viel während den Recherchen lernen, welche auch im Alltag oder im Geschäftsleben verwendet werden können. Zum Beispiel kann das Elizitieren bei Lohn- oder Vertragsverhandlungen zum eigenen Nutzen eingesetzt werden.

Phishing war als Thema für mich wesentlich interessanter, da dieses mehr technisches enthält. Ich fand es aber spannend, dass ich beim schreiben über Phishing sehr viel parallelen zum Social Engineering ziehen konnte.

% print full glossary
\glsaddall
