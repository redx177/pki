\chapter{Phishing}
Bei Phishing handelt es sich um eine Unterart des Social Engineerings. Bislang wurden stets einzelne, ausgewählte Opfer angegriffen.
Beim Phishing sucht man ein Ziel aus einer grosse Menge von möglichen Zielen ausfindig zu machen. Dazu werden oft Spam E-Mails, \Gls{glos:ircLabel} Chats oder soziale Plattformen eingesetzt. Mit denen wird versucht, eine Schwachstelle auszunutzen. Dazu werden oft präparierte Webseiten verwendet um etwas zu verkaufen oder eine sicherheitstechnische Schwachstelle auszunutzen.

Auch wenn man eine Gruppe als Ziel hat, gelten die selben Techniken wie im Kapitel \ref{sec:socialengineering} Social Engineering beschrieben.
Die Gruppe muss dazu identifiziert werden um die richtige Taktik zu wählen.

Danach müssen das Kommunikationsmodel (siehe Abschnitt \ref{fig:socialengineering:kommunikation:kommunikationsmodell} Kommunikationsmodell) gewählt werden.

Schlussendlich wird der Angriff vorbereitet und durchgeführt.

\section{Zielgruppe}
Möchte man ein Angriff vorbereiten muss man wissen, wer seine Zielgruppe ist. Die Einteilung ist wichtig, um später zu wissen wie man zum Beispiel die Spam Mails oder Kurznachrichten verfassen möchte. Möchte man Viagra-Pillen verkaufen, ist die Zielgruppe Männlich. Dann sollte auch das nachträglich erläuterte Kommunikationsmodell entsprechend wählen.

Eine Einteilung ist in folgende Bereiche möglich:
\begin{itemize}
\item Geschlecht
\item Alter
\item Interessen
\item Branche
\item etc.
\end{itemize}

\section{Kommunikationsmodell}
Wie im Abschnitt \ref{fig:socialengineering:kommunikation:kommunikationsmodell} beschrieben besteht ein Kommunikationsmodell aus folgenden Teilen:
\begin{itemize}
\item Informationsquelle
\item Übertragung
\item Kanal
\item Empfänger
\item Feedback
\item Zielort
\end{itemize}

Die \textit{Informationsquelle} ist normalerweise der Angreifer selber oder ein Auftraggeber für den Angriff.

Die \textit{Übertragung} wird vom Angreifer durchgeführt, da er die zu versendenden Nachrichten vorbereitet.

Der \textit{Kanal} ist üblicherweise eine Auswahl aus E-Mail,  \Gls{glos:ircLabel} Chats oder soziale Plattformen, da darüber eine grosse Menge von Zielen angesprochen werden können. Die Wahl ist abhängig von der zuvor gewählten Zielgruppe. Junge Leute trifft man eher in Chatrooms an. Technisch versierte Personen sind eher in E-Mail Listen zu finden.

\textit{Empfänger} ist die festgelegte Zielgruppe und der \textit{Zielort} ist oft eine präparierte Webseite, ein Download welcher getätigt werden soll, etc.

\section{Angriff vorbereiten}
Das Ziel 